\chapter{}\section{}\begin{enumerate} %Section 5.1

\item   	Can a function have more than one derivative? Can a function have more than one antiderivative?  Explain.

\item   What is the ``+ C'' thing?  What purpose does it serve? 


\item   We have second derivatives.  What about second integrals?  What would this mean?  Create some examples.  Speculate what a second integral would measure.

\item   A friend of yours has the following work on his paper.
	$$\displaylines{  \int {x^2 \left( {x^3  - 1} \right)dx}  = {{x^3 } \over 3}\left( {{{x^4 } \over 4} - x} \right) + C \cr    = {{x^7 } \over {12}} - {{x^4 } \over 3} + C \cr} $$
Gently and clearly, explain to your friend the mistake he has made and give him advice for avoiding this mistake in the future.

 


\end{enumerate}
\addtocounter{section}{2}
\section{}
\begin{enumerate}%Section 5.4

\item   Find the equation for the curve in the $xy$-plane that passes through the point $(1, -1)$ if its slope at $x$ is always $3x^2 + 2$.  \cite{FWG}

\item   You were going over the homework with a friend and saw the following work on his paper.
		$$\begin{array}{rclcrcl}
\int_{ - 1}^2 {\left( {3x - 2} \right)^5 dx} & =& {1 \over 3}\int_{ - 1}^2 {u^5 dx} &\ \ \ \ \ \ \mbox{}& u &=& 3x - 2 \cr  
  & =& \left. {{1 \over 3}\left( {u^6 } \right)} \right|_{ - 1}^2 &&  du &=& 3dx \cr  
  &  =& {1 \over 3}\left( {2^6  - \left( { - 1} \right)^6 } \right) &&  {1 \over 3}du &=& dx \cr    
  &= &21&& \cr\end{array}$$
Your friend was very happy because the answer came out to be a whole number and that is always a good sign.  You did not like having to break the news to your friend about the mistakes you found, so you gently explained the errors and how to fix them.

\end{enumerate}
\section{}
\begin{enumerate}%Section 5.5


\item   Sometimes a substitution is not always obvious.  Investigate the following possible substitutions for $$\int {{{x\,dx} \over {1 + x^4 }}} .$$	
\begin{enumerate} \item $u = 1 + x^4$\item $u = x^2$\item $u = x^3$\item $u = x^4$\end{enumerate}
Which substitution would be the best?  Provide some advice to another student for how to recognize the appropriate substitution. 

\item   Using the function $$f(x) = \int\limits_0^{x^2 } {\cos t\;dt} $$ illustrate the use of the chain rule in combination with the second Fundamental Theorem of Calculus.  To do this, first evaluate the integral in the usual way and then find $f'(x)$ in the usual way.  After that, show how to find $f'(x)$ directly from $$f(x) = \int\limits_0^{x^2 } {\cos t\;dt} $$ using the second Fundamental Theorem of Calculus.  Finally, highlight in each case where the chain rule was applied making connections between the two calculations.



\end{enumerate}






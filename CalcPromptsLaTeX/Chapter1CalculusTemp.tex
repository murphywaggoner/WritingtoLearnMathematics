\chapter{}

\section{}
\begin{enumerate}


\item Always true, sometimes true or never true:  If $a$ is a real number, then $$\sqrt {a^2 }  = a.$$


\item Explain what is meant by $(-\infty, 7]$.

\item How do you know that $${{x^2  + y^2 } \over {\left( {x + y} \right)^2 }} \ne 1? \cite{B}$$  

\end{enumerate}

\section{}
\begin{enumerate}

\item Investigate the validity of this statement:  If $f(x)$ and $g(x)$ are functions, then $$f \circ g\left( x \right) = g \circ f\left( x \right).$$


\item The absolute value of $x$ is defined to be $$
\left| x \right| = \left\{ \matrix{
  x\ \ {\rm{ for }}\ \ x \ge 0 \hfill \cr 
  x\ \ {\rm{ for }}\ \ x < 0 \hfill \cr}  \right.  .
$$
  To understand this definition, you must believe that $$\left| x \right| =  - x$$ for negative values of $x$.  Using $x = -3$ as an example, explain why $-x = (-1)x$ produces the same result as taking the absolute value of $x$.  \cite{SM}
  
  \end{enumerate}

\section{}
\begin{enumerate}



\item Explain the connection between completing the square and the quadratic formula.  

\item Is $0^r = 0$ for any value of $r$?  Try these values before making your decision:  $r = 2$, ${\textstyle{1 \over 2}},$ 1000, $-2$, $ - {\textstyle{3 \over 2}},$ 0.15 and 0.

\end{enumerate}

\section{}
\begin{enumerate}


\item a)  Solve the equation $$e^x  + e^{ - x}  = 5$$ by following these steps.  First, make a substitution $u = e^x .$  Note that $e^{ - x}  = {1 \over {e^x }}.$  Now, clear the fractions and you should recognize how to solve this equation for $u$.  Finally, now that you know the value of $u$, solve for $x$ in $u = e^x .$  (Don't forget that $e^x$ is always positive.) \\  b)  To practice your new found skill, solve $2^x  - 2^{ - x}  = 3$ for $x$.

\item Solve each of the following 2 quadratic equations (giving exact values).  Justify the methods you used to solve each equation.  $$x^2 - 2x = 0$$ $$x + x^2 = 2$$


\end{enumerate}

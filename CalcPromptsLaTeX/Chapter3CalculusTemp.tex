\chapter{}\section{}

\begin{enumerate}%Section 3.1

\item  Compare and contrast the meaning and usage of $$f'(c)$$ and $$f'(x).$$

\item  Certain values for a differentiable function are given in the following table.  Approximate the value of $f'(0.3).$  Explain how you found this approximation and why it is only an approximation.  (Maybe sketching a graph will help).
$$\begin{array}{|c|c|c|c|c|c|c|c|c|c|}\hline  x & 0.0 & 0.1 & 0.2 & 0.3 & 0.4 & 0.5 & 0.6 & 0.7 & 0.8  \cr \hline  f(x) & 5.0 & 4.1 & 4.0 & 4.6 & 5.5 & 6.2 & 6.5 & 6.1 & 5.7  \cr \hline  \end{array}$$

\item  Compare and contrast differentiability and continuity.

\item  Describe geometrically when a function typically does not have a derivative at a point.   \cite{FWG}


\end{enumerate}
\section{}
\begin{enumerate}%Section 3.2

\item  For what values of $a$ and $b$ is $2x + y = b$ tangent to $y = ax^2$ at $x = 2$?  

\item  The line $L$ is tangent to the graph of the differentiable function $y = f(x)$ at the point (5, 2).  $L$ intersects the $y$-axis at (0, 4).  Find $f'(5).$  Provide an illustration along with the writing.






\end{enumerate}
\section{}
\begin{enumerate}%Section 3.3

\item  Compare and contrast the evaluation of the derivatives of the following functions.
\begin{enumerate}

\item  $y = 3^3$ 

\item   $y = x^3$ 

\item  $y = x^x$ 

\item   $y = 3^x$\end{enumerate}

\end{enumerate}
\section{}
\begin{enumerate}%Section 3.4

\item  Using the definition of the derivative, explain why the units of velocity are distance/time and why the units of acceleration are ${\rm{distance}}/({\rm{time}})^2$.

\item  What is the difference between speed and velocity?  We use both of these words in our every day language, but one of them we usually use incorrectly.  Which one is it?  What does acceleration mean in terms of speed?  Why do we call the accelerator the accelerator in our car?  Is it always used as an accelerator in the mathematical sense of the word?  How else does acceleration happen when we are driving a car?

\item  What is the function that gives the height of an object shot straight up from the ground at an initial velocity of 30 feet/sec?  Graph this function.  Find the  velocity and acceleration functions and graph them, too.  Identify the important features of the graphs and describe what is happening at those points.  Explain why the acceleration function looks the way it does.  (What would it say about the Earth if the acceleration function looked different than it does?  Have you ever seen the movie Hypercube?)


\end{enumerate}
\section{}
\begin{enumerate}%Section 3.5

\item  A student was absent from class today and asked you to help him understand when and how to use logarithmic differentiation.  Give him several examples that do not need logarithmic differentiation and several that do.  Explain the process.

\item  A student was absent from class yesterday and asked you to help him understand the chain rule.  In particular, he needs help understanding how to find the derivatives of $\sin 5 x$ and $\sin x^5$.  Explain this to him.

\item  Compare and contrast the evaluation of the derivatives of $$e^{x\sin x} ,$$ $$e^x \sin x$$ and $$xe^{\sin x} .$$

\end{enumerate}
\section{}
\begin{enumerate}%Section 3.6

\item  Compare and contract explicit functions and implicit functions.

\item  Explain how to derive the derivative formula for $$y = \sin ^{ - 1} x.$$



\end{enumerate}
\section{}
\begin{enumerate}%Section 3.7

\item  A friend of yours missed the first day we talked about related rates problems and wants to know what we did so he can complete the homework.  Rewrite and improve Example 1 from this section of the test to help explain the process to him.  

\item  Why is it important to identify the independent variable before finding a derivative?  What is the independent variable in a related rates problem?  How does this impact how we solve a related rates problem?  

\end{enumerate}
\section{}
\begin{enumerate}%Section 3.8



\item  What is a differential?  What is it used for?  What are the notational alternatives to using a differential and why, in some cases, is a differential an advantage over the alternative?


\item  A differential helps us measure a function's sensitivity to change.  Suppose that we make a fixed small change in x, i.e., $dx$ is small and fixed.  If a small change in $x$ results in a large change in $y$, then we say the function is sensitive to change.  Since the differential gives us an approximate value for the change in $y$, then the differential also measures the sensitivity of the function.

\end{enumerate}



 





\chapter{Vectors, Lines and Planes in Space}

\label{sec:VectorsLinesAndPlanesInSpace}


\section{General}

\begin{enumerate}
\item Discuss the relationship between a 3-dimensional vector and a point in space.  \cite{EP}
\item Give an example of a real-world situation described by a triple of real numbers (other than a point in space).  \cite{EP}
\item Explain why open is not the opposite of closed?  Can a set be both open and closed?  Can a set be neither open nor closed?  Take your examples from both the line and the plane.
\item Write up to 5 distinct questions from the material in this chapter that I might ask on an exam about the following information.
$(-2, 3, 0)$ and $(1, 0, -2)$:
\item Write up to 5 distinct questions from the material in this chapter that I might ask on an exam about the following information:
$(-2, 3, 0)$, $(1, 0, -2)$ and $(1, 0, 0)$.
\item Suppose you were asked to provide a graph represented by $x = 3$.  Explain that to do so you need to know the context in which to graph (1-dimension, 2-dimensions or 3-dimensions).  Explain each possible graph and how it depicts all points where $x = 3$.
\item Look back at the homework and writing assignments you have done so far and identify concepts that you feel you know the best.  Identify areas that you need to improve on before the exam.  If you could improve on one concept before the exam, what concept would be the most beneficial to you and why? Which are the trickiest?
\item It is very important to be able to quickly and accurately visualize three-dimensional relationships.  In three dimensions, describe how many lines are perpendicular to the unit vector $\bf{i}$.  Describe all lines that are perpendicular to $\bf{i}$ and pass through the origin.  In three dimensions, describe how many planes are perpendicular to the unit vector $\bf{i}$.  Describe all planes that are perpendicular to $\bf{i}$ and that contain the origin.  \cite{SM}
\end{enumerate}

\section{Parametric Curves}

\begin{enumerate}

\item Compare and contrast curves of the form  $y = f(x)$ and parametric curves of the form $\left\{ {x = x(t),y = y(t)} \right\}$.  In particular, discuss the representations of these curves, the roles of the dependent variable, the independent variable(s) and the parameter, and the value of each representation.
\item Compare and contrast the representations of the same curve as $$y=f(x),$$ $$\left\{ {x = t,y = f(t)} \right\}$$ and $$\left\{ {x = f^{-1}(s),y = s} \right\}.$$
\item Write up to 5 distinct questions from the material in this chapter that I might ask on an exam about the following information:
$$\left\{ \matrix{ x = 3t - 5 \hfill \cr y = t^2  \hfill \cr}  \right\}$$ for $-1 \le t \le 2$.
\item Compare and contrast the graphs of $$\left\{ \matrix{  x = \cos ^2 t \hfill \cr   y = \sin ^2 t \hfill \cr}  \right\}$$ and $x + y = 1$.
\item Consider the parametric curve 
$$\left\{ \matrix{  x = a+r\cos t \hfill \cr   y = b+r\sin t \hfill \cr}  \right\}.$$  Describe as completely and as generally as possible the shape and orientation of this curve.  Describe how the curve is drawn out as $t$ ranges from 0 to $2\pi$.  Find a nonparametric representation of this curve.
\item  Consider the parametric curve 
$$\left\{ \matrix{  x = a+r t \hfill \cr   y = b+r t \hfill \cr}  \right\}$$
 for $0\le t\le t_1$.  Describe as completely and as generally as possible the shape and orientation of this curve.  Describe how the curve is drawn out as $t$ ranges from 0 to $t_1$.  Find a nonparametric representation of this curve.
\end{enumerate}

\section{Lines and Planes}

\begin{enumerate}

\item Compare and contrast the parametric and symmetric forms of the equation of a line in three dimensions.
\item Compare and contrast finding equations of lines in 2-space and in 3-space.
\item What do we need to know to find the equation of a line in 3-space?  Give as many different ways that information can be given to us.
\item Compare and contrast finding equations of lines in 2-space and equations of planes in 3-space.
\item Compare and contrast finding equations of lines in 3-space and equations of planes in 3-space.
\item Compare and contrast finding intersections of lines in 2-space and finding intersections of lines in 3-space.
\item How do we analytically describe a line in 3-space?  Why do we have to describe it that way?  Why can we not describe a line with a single equation?
\item Explain how to find the intersection and orientation of 2 lines in 3-space.  Detail the possible orientations.
\item Detail the possible orientations of a line and a plane in 3-space.  Explain how to determine which of those possible orientations a line and plane actually have.
\item Detail the possible orientations of 2 planes in 3-space.  If 2 planes intersect, what is the resulting intersection?  
\item Write up to 5 distinct questions from the material in this chapter that I might ask on an exam about the following information:
$$\left\{ \matrix{  x = 3 + 2t \hfill \cr   y = 2 - t \hfill \cr  z = 5 + t \hfill \cr}  \right\}$$ and $$\left\{ \matrix{  x =  - 3 - s \hfill \cr  y = 7 + s \hfill \cr  z = 16 + 3s \hfill \cr}  \right\}.$$
\item Write up to 5 distinct questions from the material in this chapter that I might ask on an exam about the following information.
$3x- 2y + z- 1 = 0$ and $2z + 3 = 0$.
\item Does it make sense to talk about the $x$-, $y$-, and $z$-intercepts of $$\left\{ \matrix{  x = 1 + 2t \hfill \cr   y =  - 1 - t \hfill \cr   z = 3t \hfill \cr}  \right\}?$$  Describe the method of finding where this line intersects each of the coordinate planes.  \cite{FWG}
\item Consider the planes of the form $${x \over a} + {y \over b} + {z \over c} = 1.$$  What do these planes look like in general and what do the values of a, b, and c represent?   (Hint:  Think about the lines of the form $${x \over a} + {y \over b} = 1$$ and try to generalize.)
\item Consider the plane $2x + y - z = 8$ and the line $$\left\{ {x = 1 - 2t,\;y = 2 + 5t,\;z =  - 3t} \right\}.$$  Explain how to determine the orientation of these two objects, i.e., are they parallel or do they intersect and, if they intersect, are they perpendicular?  \cite{FWG}
\item Explain how to find equations of 2 distinct planes that intersect along the line $$\left\{ \matrix{  x = 1 + t \hfill \cr y = 2 - t \hfill \cr  z = 3 + 2t \hfill \cr}  \right\}.\ \cite{FWG}$$  
\item 

\begin{enumerate} 
\item Explain how to create two different sets of parametric equations of lines that are coincident. 
\item Explain how to create two sets of parametric equations of lines that are parallel but not coincident. 
\item Explain how to create sets of parametric equations of lines that are skew. 
\item Explain how to create sets of parametric equations of lines that intersect and are perpendicular. 
\end{enumerate}

\item What is the relationship between the angle between 2 planes and the angle between the normal vectors of those planes.  Create a physical model to help illustrate this to another student.
\item How can you look at the equation of a plane and decide that it is parallel to the $x$-axis?  Generalize.
\item You and a friend were comparing homework and you got different answers to 
this question:  Find the parametric equation of the line passing through the two points $(1, 2, 2)$ 
and $(3, -1, 3)$.  $$\rm{Your}\ \rm{ answer}  \left\{ \matrix{  x = 1 + 2t \cr   y = 2 - 3t \cr   z = 2 + t \cr}  
\right\}$$$$\rm{Your}\  \rm{friend's}\ \rm{answer}\left\{ \matrix{  x =  - 2s + 3 \cr   y = 3s - 1 \cr   z =  - s + 3 \cr}  \right\}$$  Should either, both or neither of you be worried?
\item Explain how to shift back and forth between the parametric and symmetric equations of the line.  Describe one situation in which you would prefer to have parametric equations to work with and one situation in which symmetric equations would be more convenient.  \cite{SM}
\item Notice that if $c = 0$ in the general equation $ax + by + cz + d = 0$ of a plane, you have an equation that would describe a line in the $xy$-plane.  Describe how the line $ax + by + d = 0$ in the $xy$-plane relates to the plane $ax + by + d = 0$.  \cite{SM}
\end{enumerate}

\section{Vectors, Projections and Vector Products}

\begin{enumerate}

\item Compare and contrast the concepts of points in the plane and vectors in the plane. 
\item Describe how the formula for distance in 2-space is derived from the Pythagorean theorem.  Explain how to find distance in 3-space.
\item Compare and contrast the dot product and cross product including a discussion of properties and uses of the two products.
\item If $\bf{u}$, $\bf{v}$, and $\bf{w}$ are coplanar, explain why $$\bf{u} \cdot \left( {\bf{v} \times \bf{w}} \right) = 0$$ and $$\left( {\bf{u} \times \bf{v}} \right) \cdot \bf{w} = 0.$$
 \item A friend of yours missed class the day we talked about vectors for the first time.  Explain to your friend what a scalar is and how it is different than a vector.  How do you find a vector's magnitude and direction?
 \item If a vector is multiplied by a positive scalar, how is the result related to the original vector?  What if the scalar is zero?  Negative? \cite{FWG}  Less than 1?  Greater than 1?
 \item What is the dot product?  How is it calculated?  How do we interpret the dot product?  When is the dot product of two vectors equal to zero?
 \item Describe and illustrate the vector projection of a vector $\bf{u}$ onto a vector $\bf{v}$.  Give several distinct examples.
 \item What is the cross product?  How is it calculated?  How do we interpret the cross product?  When is the cross product of two vectors equal to zero?
 \item Suppose that vectors ${\bf{u}}$, ${\bf{v}}$ and ${\bf{w}}$ are coplanar, that is, if ${\bf{u}}$, ${\bf{v}}$ and ${\bf{w}}$ were positioned so they share the same inital point, then all 3 vectors would lie in the same plan.  Use a geometric argument to prove that $$
{\bf{u}} \cdot \left( {{\bf{v}} \times {\bf{w}}} \right) = 0
$$
and $$
\left( {{\bf{u}} \times {\bf{v}}} \right) \cdot {\bf{w}} = 0
.$$

 \item Write up to 5 distinct questions from the material in this chapter that I might ask on an exam about the following information:$\bf{u} = 3{\bf{i}} + 2{\bf{j}}$ and $\bf{v}= -{\bf{j}}.$
 \item Compare and contrast the usage of the words perpendicular, orthogonal and normal.
 \item What is the "right-hand rule" and what does it help us visualize?  If there were a "left-hand rule" what would it help us visualize?
 \item 
 \begin{enumerate} 
 \item Find the area of the triangle determined by the points $(1, 2, 0)$, $(-1, 2, 0)$ and $(0, 0, 0)$ using a cross product. 
 \item Check your answer to the first part of this question by finding the area using the formula $A = 0.5 bh$. 
 \item Explain why the process in the first part of this questionis useful (i.e., why would we use the cross product instead of the formula $A = 0.5 bh$?)
 \end{enumerate}
 \item We know that if $$\bf{u} \times \bf{v} = 0,$$ then $\bf{u}$ and $\bf{v}$ are parallel.  Why is this true?  Why is this not an efficient method for determining if 2 given vectors are parallel?
 \item Explain how to determine if $\bf{u} = \left\langle  a, b \right\rangle $ and $\bf{v} = \left\langle c, d\right\rangle $ are orthogonal by inspection.  Explain why we cannot decide if $\bf{u} = \left\langle a, b, c\right\rangle$ and $\bf{v} = \left\langle d, c, e\right\rangle$ are orthogonal by inspection.
 \item What is the relationship between $${\bf{a}} \cdot {\bf{b}}$$ and $${\bf{b}} \cdot {\bf{a}}?$$  What is the relationship between $${\bf{a}} \times {\bf{b}}$$ and $${\bf{b}} \times {\bf{a}}?$$  Be sure to use geometry as part of your explanation.
 \item Compare and contrast $${{\left| {{\bf{a}} \cdot {\bf{b}}} \right|} \over {\left| {\bf{a}} \right|\left| {\bf{b}} \right|}}$$and $${{\left| {{\bf{a}} \times {\bf{b}}} \right|} \over {\left| {\bf{a}} \right|\left| {\bf{b}} \right|}}.$$
 \item What is $$\vec 0 \cdot \left( {a\hat i + b\hat j + c\hat k} \right)?$$  How should we interpret this result?  
 \item If $(a, b)$ and $$\left\langle {a,b} \right\rangle  = a\hat i + b\hat j$$ are both represented by an ordered pair of numbers, why can we talk about the length of $$\left\langle {a,b} \right\rangle $$ but not $(a, b)$?
 \item Compare and contrast scalar multiplication (a)(b) and the dot product $$\bf{u} \cdot \bf{v}.$$
 \item Compare and contrast scalar multiplication $(a)(b)$ and the cross product $$\bf{u} \times \bf{v}.$$
 \item Show that except in degenerate cases, $$\left( {\bf{u} \times \bf{v}} \right) \times \bf{w}$$ lies in the plane of $\bf{u}$ and $\bf{v}$ whereas $$\bf{u} \times \left( {\bf{v} \times \bf{w}} \right)$$ lies in the plane of $\bf{v}$ and $\bf{w}$.  Describe the degenerate cases.  \cite{FWG}
 \item Prove that $$\left| {\bf{u} \cdot \bf{v}} \right| \le \left\| \bf{u} \right\|\,\left\| \bf{v} \right\|.$$ Under what circumstance is $$\left| {\bf{u} \cdot \bf{v}} \right| = \left\| \bf{u} \right\|\,\left\| \bf{v} \right\|? \cite{FWG}$$  
 \item A river 2.1 miles wide flows south with a current of 3.1 miles per hour.  What speed and heading should a motorboat assume to travel across the river from east to west in 30 minutes?  \cite{SBS}
 \item If $\bf{u}$ and $\bf{v}$ are nonzero vectors and $$\left\| \bf{u} \right\| = \left\| \bf{v} \right\|,$$ does it follow that $\bf{u}=\bf{v}$?  \cite{SBS}
 \item What is the standard basis of vectors in ${\Re}^3 $?  What purpose does the bases serve?
 
 \item Assume that $\bf{u}$ and $\bf{v}$ are 2-dimensional vectors.  Which of the following operations are not defined and why?  If the operation is defined, is the result scalar or vector? 
 
 \begin{center}
 \begin{tabular}{l}
 $\bf{u} \cdot \bf{v}$\\ 
 $\bf{u} \times \bf{v}$\\
 $\bf{u} \cdot \left( {\bf{v} \cdot \bf{w}} \right)$\\ 
 $\bf{u} \times \left( {\bf{v} \times \bf{w}} \right)$\\ 
 $\bf{u} \cdot \left( {\bf{v} \times \bf{w}} \right)$\\
 $\bf{u} \times \left( {\bf{v} \cdot \bf{w}} \right)$\\ 
 \end{tabular} 
 \end{center}
 
 Repeat this exercise assuming that u and v are 3-dimensional vectors.  Discuss.
 
 \item Briefly describe each of the following processes:  how to determine if 2 vectors are parallel, how to determine if 2 vectors are perpendicular, how to determine if 3 points are collinear, andhow to determine if 4 points are coplanar.  \cite{SM}
 \item Briefly describe how to construct a vector parallel to a given vector.  Briefly describe how to construct a vector perpendicular to a given vector.  Given a vector, describe how to construct 2 other vectors such that the three vectors are mutually perpendicular.  \cite{SM}
 
 \item Consider 2 3-dimensional vectors $\bf{u}$ and $\bf{v}$.  Describe the results of each of the following vector expressions in words and with a diagram where possible.
 
 \begin{center}
 \begin{tabular}{l}
 $3\bf{v}$\\ 
 $-3\bf{v}$\\
 
  $\left\| \bf{u}\right\| + \left\| \bf{v}\right\|$\\ 
  $\bf{u} \cdot \bf{v}$\\
  $\bf{v} \cdot \bf{u}$\\
  $\bf{v} \cdot \bf{v}$\\
 $\bf{u} \times \bf{v}$\\
 $\bf{v} \times \bf{u}$\\
  $\bf{v} \times \bf{v}$\\
	${{\bf{v}} \over \left\| {\bf{v}} \right\|}$\\ 
  \end{tabular} 
 \end{center}
 

\end{enumerate}

\section{General Surfaces}

\begin{enumerate}

 \item Compare and contrast finding equations of circles in 2-space and spheres in 3-space.
 \item In space why do we refer to the surfaces like $x^2 - y = 0$ as a cylinder.  Explain the geometry of this and related surfaces.  How do we distinguish cylinders from other surfaces?  How do we graph cylinders?
\end{enumerate}
